
%-------------------------------------------------------------------%
%--
%-- Type rules and judgements
%--
%-------------------------------------------------------------------%

%style parameters
\newlength{\typerulevspace}
\setlength{\typerulevspace}{0cm}
\newlength{\typerulehorskip}
\setlength{\typerulehorskip}{1cm}
\newlength{\judgementhspace}
\setlength{\judgementhspace}{0.1cm}
\newcommand{\typerulespacing}{1.15}
\newcommand{\typefigurespacing}{1.15}
\newcommand{\spacebox}[1]{\mbox{\rule[-.3cm]{0cm}{0.8cm}$#1$}}
\newcommand{\smalllabel}[1]{\scriptsize{\hspace{-1mm}#1}}

% arg 1: formatting: {b,n,r}
% arg 2: type rule name
% arg 3: premises
% arg 4: conclusion
\newcommand{\formattedtyperule}[4]{\ovr[#1]{\renewcommand{\arraystretch}{\typerulespacing}\begin{array}{c}#3\end{array}}[#2]{\renewcommand{\arraystretch}{\typerulespacing}\begin{array}{c}#4\end{array}}}

% arg 1: type rule name
% arg 2: premises
% arg 3: conclusion
\newcommand{\typerule}[3]{\formattedtyperule{n}{#1}{#2}{#3}}

\newcommand{\typerulenewline}{\\[\typerulevspace]}       % continue on the next line
\newcommand{\typeruledoublenl}{\\[3mm]}
\newcommand{\typeruleskip}{\hspace{\typerulehorskip}}    % skip some horizontal space (for next item)
\newcommand{\typerulesmallskip}{\hspace{.5\typerulehorskip}}

\newcommand{\typerulename}[1]{\mbox{\textsc{(#1)}}}
\newcommand{\typerulenamewithset}[2]{\typerulename{#1-#2}}
\newcommand{\typerulenamewithsetplus}[3]{\typerulename{#1-#2}^{\mbox{$#3$}}}

% use renewcommand to specify the _current_ type rule set. 
% this will appear in the turnstyles/judgements 
\newcommand{\typeruleSet}{}     %subscript
\newcommand{\typeruleSubset}{}  %superscript

\newcommand{\turnstyle}{\mbox{$\vdash\!\!\!_{\typeruleSet}^{\raisebox{0.05cm}{\typeruleSubset}}$}}
\newcommand{\turnstyleOne}[1]{\ {\renewcommand{\typeruleSet}{#1}\turnstyle}\ }
\newcommand{\turnstyleTwo}[2]{\ {\renewcommand{\typeruleSet}{#1}\renewcommand{\typeruleSubset}{#2}\turnstyle}\ }

\newif\ifprotectjudgements
\protectjudgementstrue

\newcommand{\judgement}[3]{\ifprotectjudgements\mbox{$\unprotectedjudgement{#1}{#2}{#3}$}\else\unprotectedjudgement{#1}{#2}{#3}\fi}
\newcommand{\unprotectedjudgement}[3]{#1 \hspace*{\judgementhspace}\turnstyle\hspace*{\judgementhspace} #2 : #3}

% named entailment relation
% #1 name of rule
% #2 premisses
% #3 consequences
\newcommand{\entailN}[3]{\entail{#2}{#3} ~~~ #1}

% unnamed entailment relation
\newcommand{\entail}[2]{#1 ~ \Vdash ~ #2}


       
